\documentclass[11pt]{article}
\usepackage{amsfonts}
\usepackage[margin=1in]{geometry}
\usepackage{amsmath}
\title{HomeWork 5}
\author{Ankit Sompura}
\newcommand\Inn{%
  \mathrel{\ooalign{$\subset$\cr\hfil\scalebox{0.8}[1]{$=$}\hfil\cr}}%
}
\begin{document}

\maketitle

\section{Complexity Analysis}


	1.	\indent def doNothing(someList):\\
			\indent \indent \indent return False\\\\
			Answer: O(1)\\\\
	2. 	\indent def do Something(someList):\\
	\indent \indent \indent if len(someList)$ == 0:$ \\
	\indent \indent \indent \indent return $0$\\
	\indent \indent \indent else if len(list $ == 1):$\\
	\indent \indent \indent \indent return $1$\\
	\indent \indent \indent else:\\
	\indent \indent \indent \indent return doSomething(someList[1:])\\\\
	Answer: O(n)\\\\

	3. \indent def doSomethingElse(someList):\\
			\indent \indent \indent \indent  n = len(someList)\\
			\indent \indent \indent \indent for i in range(n):\\
			\indent \indent \indent \indent \indent for j in range(n):\\
			\indent \indent \indent \indent \indent \indent if someList[1] $>$
			 someList[j]:\\
			\indent \indent \indent \indent \indent \indent \indent temp = someList[1]\\
			\indent \indent \indent \indent \indent \indent \indent someList[i] = someList[j] \\
			\indent \indent \indent \indent \indent \indent \indent someList[j] = temp\\
			\indent \indent \indent \indent return someList \\
			Answer: $O(n^2)$\\\


\section{Order of Complexity}

	1. $f(n) = 3n + 2 \in O(n)$\\\\
	Answer:\\
	$ c * n  \geq  3n + 2  \hspace{.5cm} \forall n_0 $\\
	$2n + 3n \geq 3n+2  \ \forall n > 0 $\\
	$ 2n \geq 2$\\
	$n \geq 1$\\
inequality holds
	2. $g(n) = 7 \in O(1)$\\\\
	Answer:\\
	g(n) = $ 7 \in O(1)$\\
	$ c * n \geq 7$\\
	$ 7(n) \geq 7$\\
	$ n \geq 1$\\

	3. $h(n) = n^2 + 2n + 4 \in O(n^2)$\\\\
	Answer:\\
h(n) = $ n^{2} + 2n + 4 \in O(n^{2})$\\
$c * n^{2} \geq N^{2} +2n + 4 \ \forall > n_0 $\\
$ 7n^{2} \geq n^{2} + 2n + 4 $\\
$ 6n^{2} + n^{2} \geq n^{2} + 2n + 4  \  \forall n > 1$\\
$ 6n^{2} \geq 2n + 4 $\\
$  n^{2} \geq  \frac{1}{3n} + \frac{2}{3}$\\


\section{Mathematical Induction}


1. $ 1 + 2 + 3 + ... + n = \frac{n(n+1)}{2}$\\\\
		Answer:\\
base case
$1+2+3+...+n = \frac{n(n+1)}{2}$\\
$n=1$\\
$1 = \frac{1(1+1)}{2}$\\
$1 = 1$\\
is true\\
inductive hypothesis \\
assume n = k\\
$1+2+3+...+k = \frac{k(k+1)}{2}$\\
$1+2+3+...+ k + k+1 = \frac{k+1( k+1+1)}{2}$\\\
$ \frac{k(k+1)}{2} + k+1 = \frac{k^2+3k+2}{2}  $\\\\
$ \frac{k^{2}  }{2} + \frac{k}{2} + k +1 = \frac{k^{2}}{2} + \frac{3k}{2} +1 $\\\\
$ \frac{k^{2}}{2} + \frac{3k}{2} +1  = \frac{k^{2}}{2} + \frac{3k}{2} + 1 $\\\\
is true\\\\



2. $ 2 + 2^2 + 2^3 + 2^4 + ... + 2^n = $
	$2^(n+1)$
	$ - 2$\\\\
Answer:\\
base case\\
$ 2n^{2} + 2n^{3} + 2n^{4}  $ ... $ + 2n^{n}$ = $ 2n^{n+1} -2  $\\
$2+2^{2}+2^{3}+2^{4}+ ... + 2^{n} = 2^{n+1} -2$\\\
$n=1$\\
$2^{1} = 2^{1+1} -2$\\\
$ 2 = 2^{2} -2$\\\
$ 2 = 4-2 $\\\
$2 =2 $\\
is true\\
inductive hypothesis\\ 
assume n = k \\
$ 2+2^{2}+2^{3}+2^{4}+2^{k} = 2^{k+1} -2 $\\
$ 2^{k+1} -2 + 2^{k+1 +1} -2$\\\\
$ 2^{k+1} -2 +2^{k+1} = 2^{k+2}-2$\\\\
$ 2^{k+2} -2 = 2^{k+2} -2 $\\\\
is true





















\end{document}